%
% Definições de comandos que são utilizados no modelo para facilitar demonstrações
%
%
\newcommand{\urlmodelosimples}{https://www.overleaf.com/project/58a3a66af9bb74023ba1bd56}
\newcommand{\urlmodelo}{\url{\urlmodelosimples}}

\newcommand{\explicacao}[1]{\todo[nolist,inline,color=yellow]{#1}}
\newcommand{\explicacaoErro}[1]{\todo[nolist,inline,color=VioletRed]{ERRADO: #1}}

\newcommand{\mostraComandoLaTeX}[1]{\textbf{\textit{$\backslash$#1}}\index{#1}}
\newcommand{\mostraPacoteLaTeX}[1]{\textbf{\textit{#1}}\index{#1}}

\newcommand{\textoFalso}[2]{\textcolor{RawSienna}{\textit{\textbf{#1} \lipsum[#2]}}}
\newcommand{\preencheComTexto}{\textoFalso{Texto para preenchimento do espaço, simulando texto real. Esse texto está com cor diferente para facilitar a identificação, mas esse tipo de mudança de cor no texto não deve ser utilizado no trabalho.}{3-5}}

% Para facilitar mudanças no site a página 404 tem um redirecionamento pelas chaves
% utiliza diretamente esse sistema de redirecionamento
\newcommand{\dicasIvan}[1]{\href{https://dicas.ivanfm.com/404.html?key=#1}{https://dicas.ivanfm.com/#1}}