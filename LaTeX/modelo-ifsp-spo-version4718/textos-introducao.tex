
% ----------------------------------------------------------
% Introdução
% ----------------------------------------------------------

\chapter[Introdução]{Introdução}
    Os estabelecimentos veterinários estão sujeitos a rigorosas leis e fiscalização por parte dos órgãos de classe (\acs{crmvs} e \acs{cfmv}). São exigidos documentos comprobatórios dos atendimentos, procedimentos e medicações utilizadas, que devem ser guardados por um prazo de 2 a 20 anos. Tais documentos são usados tanto para fins de fiscalização, como para fins jurídicos, em caso de processos, assim como devem estar disponíveis para o tutor do animal.
    
    Prontuários veterinários em formato digital são oferecidos por diversos sistemas de gerenciamento de clínicas veterinárias, e são amplamente usados como forma de garantir um backup dos prontuários de forma simples, segura e efetiva.
    Apesar disso, as versões digitais não podem ser utilizadas como documentos oficiais em caso de provas judiciais. 
    
    A gestão de medicamentos controlados (de uso restrito e com retenção de receitas, de acordo com as leis vigentes da INSTRUÇÃO NORMATIVA Nº 35, DE 11 DE SETEMBRO DE 2017, Capítulo I, art 2º, § IV, Capitulo IV, § 11  \cite{normativa} é feita utilizando um caderno, de capa dura, no formato brochura, onde o médico veterinário responsável técnico (RT) anota as datas de entrada dos medicamentos no estoque e quanto desse medicamento foi utilizado no dia, para cada procedimento. Este caderno é denominado Livro-Registro.


\section{Análise da situação atual}
    
\section{Problema}
    Assumindo as seguintes premissas:

    \begin{itemize}
        \item 
        Os sistemas de prontuários digitais existentes no mercado permitem a edição e alteração dos dados inseridos, sem garantir um histórico dessas alterações, ou meios de rastrear as alterações, em caso de fiscalização por órgão públicos ou perícias judiciais.
        
        \item
        Os arquivos são armazenados de forma física (utilizando pastas-fichários) e são carimbados e assinados pelo médico veterinário.
        
        \item
        O gerenciamento dos medicamentos controlados feitos em livro-registro demandam um tempo extra, gasto pelo profissional, que deve fazer as contas manualmente para cada valor, muitas vezes tendo que anotar em diversos documentos os mesmos dados.
        
        \item
        O armazenamento desses arquivos ocupa espaço e estão sujeitos a danos e perdas por mau armazenamento (umidade, fogo, roubo, etc).
        
        \item
        O preenchimento desses dados em versões físicas e depois transpostos para versões digitais é demorado e sujeito a falhas, demandando um tempo que poderia ser melhor empregado para os envolvidos.
        
        \item
        A consulta a esses documentos pode ser prejudicada caso não haja uma boa organização por parte dos responsáveis pelo estabelecimento.
    \end{itemize}


    Partindo dos fatos relacionados, como armazenar informações de dados documentais virtuais em clínicas veterinárias que possam ser aceitas como documentos oficiais pelo órgão de classe?

\section{Objetivos}
    \explicacao{Em muitos casos uma seção única de objetivos é suficiente}
    \preencheComTexto
    
    \subsection{Objetivo Principal}
    \preencheComTexto
    
    \subsection{Objetivos Secundários}
    \preencheComTexto
    
    \section{Justificativa}
    \preencheComTexto
    
    \section{Estrutura do Estudo}
    \preencheComTexto

