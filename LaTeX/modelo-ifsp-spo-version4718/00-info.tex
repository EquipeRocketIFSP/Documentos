
\explicacao{2022  overleaf tem compilado em modo rascunho/draft na visualização rápida principalmente para usuarios não logados. No modo rascunho os links não são gerados e outros elementos não são apresentados corretamente, observe no menu lateral da opção \enquote{Recompile} para escolher a opção correta de compilação \url{https://pt.overleaf.com/blog/642-tip-of-the-week-overleaf-v2-autocompile}}

Esse documento foi feito a partir do modelo canônico de trabalho acadêmico da classe \abnTeX, o acesso projeto com fontes e ao \acs{pdf} pode ser feito em 
\urlmodelo. Esse modelo foi feito como exemplo para alunos dos cursos de informática do \ac{ifsp} campus São Paulo. Um modelo para apresentações (\textit{slides}) utilizando Beamer está disponível em  \url{https://www.overleaf.com/read/qjrjhqwqbqqw}


Este documento não pode ser considerado como um padrão a ser seguido em sua totalidade, ele tem como maior objetivo demonstrar como utilizar o \LaTeX\ para obter um documento atendendo ao máximo o padrão do \ac{ifsp} e \ac{abnt}. Ele não foi montado como um curso de {\LaTeX} já que existem diversos disponíveis na internet. O formato textual está mais próximo de um manual do que a um trabalho acadêmico.

Esse modelo é atualizado constantemente tentando apresentar e resolver problemas que aparecem nos trabalhos das disciplinas. Se você encontrar um problema ou inconsistência envie a informação para o seu professor de informática do \ac{ifsp} campus São Paulo. Portanto é importante sempre utilizar a ultima versão dos arquivos de classe (\textbf{.cls}) deste modelo em seu documento de forma a utilizar todos os ajustes e configurações aplicados nesse documento.

O formato geral de cada trabalho a ser desenvolvido depende do contexto, mas os principais capítulos de todos os trabalhos são : Introdução, Revisão de Literatura, em seguida os capítulos referentes ao desenvolvimento do trabalho e finalmente as Considerações Finais.

Para entender corretamente como desenvolver seu documento em {\LaTeX} é importante fazer uma leitura dos arquivos fonte {\LaTeX} e não somente do documento \acs{pdf} gerado pelo compilador {\LaTeX}. E fazer também leitura das definições de referências (arquivos \textbf{.bib}).

Algumas bibliotecas \LaTeX\ disponíveis no overleaf estão desatualizadas, para melhores resultados é recomendável a utilização de outro compilador em seu computador utilizando as ultimas versões de todas bibliotecas. Existem diversas opções como o \href{https://www.tug.org/texlive/}{TeX Live}  e o \LaTeX\ faz parte de diversas distribuições Linux. Além disso o \textbf{latexdiff} que é muito útil para reduzir o texto gasto verificando mudanças entre versões não faz parte do overleaf.

Esse documento possui elementos apresentados em cores diferentes, isso serve para demonstração de situações especificas, mas em um documento real isso deve ser evitado, mantendo o texto geral na cor preta padrão.

Leia com cuidado :
\begin{itemize}
    \item \dicasIvan{textos};
    
    \item exemplos de \LaTeX \space no \autoref{cap-exemplos};

    \item Cuidado para não cometer os erros indicados no \autoref{erros-comuns-capitulo} e \autoref{erros-projetos};
    
    \item Revisão de Textos no \autoref{revisao-de-textos};

    \item \autoref{elementos-nao-textuais} sobre elementos não textuais que fala sobre o maior problema dos alunos que é de tentar posicionar as ilustrações.
\end{itemize}


Esse modelo foi feito inicialmente com o \textbf{abntex2cite} e atualmente está utilizando \textbf{biblatex-abnt}. Apesar disso alguns comandos que funcionam somente no \textbf{abntex2cite} permanecem no documento para exemplificar o funcionamento.
%\todo[inline]{migrar do abntex2cite para biblatex-abnt \url{http://www.abntex.net.br/\#abntex3-e-biblatex-abnt}}


\noindent\hrulefill

\newpage
